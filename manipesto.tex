% a4 paper size, 11pt font
% scartcl is the KOMA-script article class, which lets you do things like have a
% subtitle. With some configuration to make headings and such be in a serif
% computer modern font, like a normal article.
\documentclass[a4paper,11pt,headings=standardclasses,parskip=half]{scrartcl}
\author{Stjin van Dongen et al.}
\title{Manipesto}
\subtitle{The Complete and Unabridged Rules of Pesten}

\usepackage{mystyle}

% Mostly a faithful transcription of Stijn van Dongen's original manuscript

\begin{document}
 \maketitle\thispagestyle{empty} % no page number under title

 \section{Introduction}

 The first step to enlightenment is to realise that Uno is one of the biggest
 scams ever to have been perpetrated in the history of tabletop games. The
 Dutch, being exceptionally stingy by nature, realised this before Uno even
 existed, and invented the game known as ``Pesten'', which requires only a
 normal deck of cards. The name is a verb translating to something like
 ``pestering'', ``antagonising'', or ``bullying'', which is (as in Uno) the aim
 of the game.

 It is in the family of ``Crazy Eights''-like card games
 (see \url{https://en.wikipedia.org/wiki/Crazy_Eights}).

 This document lays out the rules and regulations for Pesten per the van Dongen
 house rules.

 \section{Basic Play}

 This is a game between two teams of equal size, so an even number of players is
 needed. \emph{It is fun to play a best-of-seven match,} where the first team to
 reach four points wins. Play best-of-five or three depending on time available.
 \emph{In this description of the rules the parts in boldface convey the most
 important information.} The rules are a little bit similar to Uno, with quite a
 few differences and additions.

 The game can be played without teams with any number of players, although team
 play is more fun. In a two-player game the ten-rule and the King-rule have no
 effect.

 The game is normally played with two decks of normal cards, including Jokers,
 although it works with any nonzero multiple of one as the number of decks.

 \emph{Pesten is played with an even number of players arranged in a circle.}
 The players are split into two teams, where \emph{players from the two teams
 alternate positions.} If the teams are called A and B order of play is thus
 a-b-a-b (et cetera). A team wins a round if one of its players is the first to
 discard all their cards according to the rules. \emph{Players in a team work
 together;} this is achieved by obstructing players from the other team and
 supporting players in your own team. \emph{You are allowed to discuss (briefly)
 theories about what suit someone in the opposing team may require if they are
 close to going out. You are not allowed to discuss your own cards.}

 Someone (the dealer) shuffles the cards and deals the cards until every player
 has a hand of 7 cards. The dealer then turns a card, this is the first card of
 the discard pile. \emph{If this card has special meaning it takes effect
 immediately.} As usual, the role of dealer shifts to the next player
 (clockwise) after every round. Normally after the first card is turned it is
 the turn of the player sitting to the left of the dealer, unless the card is
 one of 7, 8, 10, or King (see below). If the card is a Jack the dealer has to
 call the suit first (see below).

 \emph{When discarding, the card played has to match the card on top of the
 discard pile, either by its suit or by its rank.} This means either the rank
 (or face) is the same, or the suit is the same. When a Jack is on top of the
 discard pile someone will have called the suit; in this case a match of suit
 has to be with the one called. A Joker can be played on any card, unless a
 two-sequence is active (see below).

 \emph{If it is your turn and you can discard then you have to discard,} even
 if your only option is a card you would like to save for later, such as a
 Joker. If you cannot discard, you have to take a card from the pile. If this
 new card cannot be played then your turn is over. If it can be played you have
 to do this. Additionally, if you are then able to discard additional cards by
 playing a set (see below), you are allowed but not required to do this.

 \emph{If you are down to one card you have to knock on the table and say 'Last
 card'.} This is always required, including when you discard a set, or if you
 pick up from the pile and play the new card. You have to do this before it is
 the next player's turn. The next player is not supposed to suddenly up the pace
 of the game to trick someone into being too slow; be gracious. Should you have
 forgotten to knock or say 'Last card' you are given two penalty cards from the
 pile. It may happen that you have to knock and say 'Last card' more than one
 time in a single turn (this will involve playing a seven and picking up a card
 from the pile).

 \section{Meaning of the cards}

 % a bit of hackery to put paragraphs inside a table. most of the parskip stuff
 % is redundant but it means that if you want to add extra paragraphs, it's not
 % a hassle.
 % TODO: really this bit could be made a lot more DRY
 \newlength{\currentparskip}
 \setlength{\currentparskip}{\parskip}
 \begin{longtable}{rl}
  \toprule
  \bfseries Card & \bfseries Meaning \\
  \midrule
  \endhead
  2 & \begin{minipage}[t]{0.9\columnwidth}%
  \setlength{\parskip}{\currentparskip}
   \emph{Take two.} A 2 can be the start of a \emph{two-sequence.} If player
   \(A\) plays a two then the next player \(B\) has to take two cards from the
   pile, unless \(B\) can follow up and start a two-sequence. Following up can
   happen by playing (I) any other 2 or (II) a 3 of the same suit. Then player
   \(C\) has to take four (case I) or five (case II) cards, unless player \(C\)
   can follow up again. This continues.

   \emph{A two-sequence can always be extended by playing a card with the same
   rank, or a card with the same suit but exactly one rank higher.} Once a
   player \(X\) cannot extend the two-sequence, \emph{all ranks in the sequence
   are summed and \(X\) has to take that number of cards from the pile and it is
   the next player's turn.} In very rare cases, that are to be celebrated and
   woven into tapestries, this may total 20, 30 or more cards.

   \begin{itemize}
    \item Sets (see below) are allowed in a two-sequence, refer to examples.
    \item Only the last card in a set contributes to the total summed rank.
    \item Should you get this far, in a two-sequence cards lose any other
          special meaning, for example the seven is no longer sticky and the
          eight does not wait.
   \end{itemize}

   It is common practice to keep low ranks such as three and four in your hand
   for as long as possible as they may be put to use in a two-sequence. Should
   someone discard a three, it may indicate they have few or no cards left of
   that suit. \emph{You are allowed to share such knowledge with your
   team-mates.}
  \end{minipage} \vspace{0.5em} \\
  7 & \begin{minipage}[t]{0.9\columnwidth}%
   \setlength{\parskip}{\currentparskip}
   \emph{Sticky seven. If you discard a seven it is still your turn} and you
   need to discard again or pick up from the pile. The seven you played is in
   immediate effect as the top card, so you need to play another seven, a card
   of matching suit, or a joker.
  \end{minipage} \vspace{0.5em} \\
  8 & \begin{minipage}[t]{0.9\columnwidth}%
   \setlength{\parskip}{\currentparskip}
   \emph{Eight wait. The next player is skipped and foregoes their turn.}
  \end{minipage} \vspace{0.5em} \\
  10 & \begin{minipage}[t]{0.9\columnwidth}%
   \setlength{\parskip}{\currentparskip}
   \emph{Teen machine (washing cycle) / ten and again.} \\
   \emph{Play takes one step back.} Suppose player \(T\) discards a ten. The
   player \(B\) next to \(T\), against the direction of play, gets a turn. This
   will usually be the player who just had a turn before \(T\) did (unless an 8
   was played). After player \(B\) it will usually be \(T\)'s turn again, unless
   \(B\) discards an 8, a 10 or a King. \emph{When 10, 8 and Kings are played
   consecutively the potential for confusion is great.} Pay attention! If a 10
   is played on a 10, play takes another step back in the same direction as the
   first step back. If a King is played on a ten, play takes another step back
   in the same direction as the first step back and additionally play continues
   in that direction.
  \end{minipage} \vspace{0.5em} \\
  J & \begin{minipage}[t]{0.9\columnwidth}%
   \setlength{\parskip}{\currentparskip}
   \emph{Jack (of all trades/suits).} \\
   The player discarding the Jack has to call the suit that needs to be matched.
   The next player can either play any card of the called suit, another Jack, or
   a Joker.
  \end{minipage} \vspace{0.5em} \\
  K & \begin{minipage}[t]{0.9\columnwidth}%
   \setlength{\parskip}{\currentparskip}
   \emph{King fling / Reversking.} \\
   \emph{The direction of play is changed.} If the current direction is
   \(A\)-\(B\)-\(C\) and after \(A\)'s turn \(B\) discards a King, then it is
   \(A\)'s turn again and play
   continues in that direction. Direction changes from clockwise to
   counter-clockwise or vice versa.
  \end{minipage} \vspace{0.5em} \\
  \(\ast\) & \begin{minipage}[t]{0.9\columnwidth}%
   \setlength{\parskip}{\currentparskip}
   \emph{Joker take five.} \\
   \emph{If a joker is discarded by player \(A\), the next player \(B\) has to
   take five cards,} unless \(B\) has a Joker - this Joker must be played and
   the next player \(C\) has to take ten cards, unless \(C\) has a joker, and so
   on. \emph{In short, Jokers stack. Jokers cannot be played in a two-sequence,
   and 2 cannot be played as a response to a Joker.} It is possible to play a
   set of Jokers (see below). The first player without a Joker has to pick up
   the total from the pile. After this the same player must declare the new suit
   (but cannot discard), and it is the next player's turn.
  \end{minipage} \vspace{0.5em} \\
  \bottomrule
  \caption{Card meanings \label{tab_card_meanings}}
 \end{longtable}

 \section{Sets}

 \emph{In every situation it is possible, if you have the right cards, to
 discard a completing set or a new set.} In the first case (completing) you use
 the top card on the discard pile and it will be part of the set. A set consists
 of \emph{at least three cards of the same rank,} or of at \emph{least three
 consecutive cards of the same suit} (either of increasing or decreasing rank).
 If you complete a set using the top card of the discard pile then that card is
 part of the set and the set is formed by discarding two or more cards. In the
 second case, when discarding a new set, the first card in the set has to fit on
 top of the pile according to the normal rules and you discard at least three
 cards. A set may consist of more than three cards, as long as they are either
 of the same rank or strictly consecutive of the same suit.

 \emph{Only the last card in a set has its special meaning (if any), other cards
 in the set lose any special meaning.} The last card in a set will be the card
 on top of the discard pile.

 \begin{itemize}
  \item \emph{When discarding a new set, the first discarded card has to fit on
        top of the pile according to the normal rules.} For example, on ♥5 you
        can discard ♥10 ♥J ♥Q, or ♥10 ♦10 ♣10, or ♣5 ♣4 ♣3, or ♠5 ♠6 ♠7 ♠8. In
        the first case the last discarded card (♥Q) has no special meaning and
        ♥10 loses its special meaning. If you discard in decreasing order, ♥Q ♥J
        ♥10 the ♥10 retains it special meaning. \emph{When completing} you could
        add to the ♥5 on top of the file, for example ♥4 ♥3, ♦5 ♣5, or ♥6 ♥7.
  \item \emph{If the last card in your set is 7 it is still your turn.} The next
        card you play can be any card normally allowed after a 7, this could
        even be the start of another set. For example following ♥5 ♥6 ♥7 you
        could discard ♥A ♣A ♦A or ♥10 ♥J ♥Q or ♥Q ♥J ♥10.
  \item \emph{Sets with cards of the same rank can have repeated suits,} thus ♥6
        ♥6 ♠6 is fine.
  \item \emph{It is possible to discard a set in a two-sequence.} In summary:
        the set can either be a completing set or a new set and \emph{the result
        should fit in the two-sequence (no gap allowed in the ranks)}. Suppose
        on top of the pile is a card with rank \(V\) and suit \(S\) and we are
        currently in a two-sequence. The first card in your set has to fit in
        the sequence, so either it has the same rank \(V\) or it has rank \(V +
        1\) and the same suit \(S\). Your set can take various forms:
        \begin{itemize}
         \item all with the same rank \(V\) (two or more cards).
         \item all with rank \(V + 1\) (three or more cards) if the first has
               suit \(S\). consecutive, increasing and of identical suit if the
               first card has rank \(V\).
         \item consecutive, increasing and of the same suit \(S\) if the first
               card has rank \(V + 1\).
        \end{itemize} Only the last card in the set contributes towards the
        total of the two-sequence. See also examples further below.
  \item \emph{In theory it is allowed to put two or more Jokers on a Joker to
        complete a set, or to discard three or more Jokers as a set.} In
        practice this opportunity rarely arises and it may be unwise to utilise
        should it happen. Only the last Joker is special, so the set adds 5 to
        the current tally. We have never experienced a game where a team won by
        discarding two or more jokers, \emph{please send a witness report in the
        form of a tapestry if this happens.}
  \item If a Jack is discarded and a suit is called, then J Q K or J 10 9 are
        valid sets regardless of the set suit.
  \item \emph{As in other card games, Ace can perform the role of both 1 and
        14.} Hence 3 2 A and A 2 3 are valid sets. It is possible to continue
        beyond from K to A to 2 or the other way round.
 \end{itemize}

 \section{Examples of sets in play}

 ♥2 is on top of the pile as part of a two-sequence. Some examples of continuing
 play with a set:

 \begin{itemize}
  \item Two or more 2s. This contributes only two towards the two-sequence
        total.
  \item ♥3 ♥4 (a completing set for ♥2). This contributes four towards the
        total.
  \item ♥3 ♥4 ♥5 - contributes five towards the total.
  \item ♥3 ♥4 ♥5 ♥6 - contributes six, and so on.
  \item ♦2 ♦3 ♦4 (a new set). Contributes four.
  \item ♥3 ♣3 ♦3 (a new set). Contributes three.
  \item ♥4 ♥5 ♥6 is not allowed because of the gap between ♥2 and ♥4.
 \end{itemize}

 ♥6 is on top of the pile and is not part of a two-sequence. This gives more
 options, as sets of consecutive cards can be decreasing, and new sets have no
 restriction on gaps as in the case of two-sequences (only options that would
 not be allowed in a two-sequence are shown):

 \begin{itemize}
  \item Three or more 9s provided the first card is ♥9 (mind the gap)
  \item ♥Q ♥K ♥A (gap)
  \item ♥5 ♥4 (decreasing)
  \item ♠6 ♠5 ♠4 (decreasing)
 \end{itemize}

 Happy Pesten!
\end{document}
